%----------------------------------------------------------------------------------------
% Requisiti
%----------------------------------------------------------------------------------------

\documentclass[10pt]{softeng} % Document font size and equations flushed left

%----------------------------------------------------------------------------------------
%	DOCUMENT INFORMATION
%----------------------------------------------------------------------------------------

\include{current-phase}

\DocumentTitle{Use Case Model} % Document title

%----------------------------------------------------------------------------------------

\begin{document}

\startofdocument{}

\section{Introduzione}

Di seguito \`e illustrata la traduzione dei requisiti funzionali del sistema in casi d'uso.
I casi d'usi forniscono una descrizione delle funzioni e dei servizi offerti dal sistema di Home Banking dal punto di vista degli attori che interagiranno con il sistema.

Il documento illustra il modello dei casi d'uso per mezzo di:
\begin{itemize}
	\item diagrammi dei casi d'uso;
	\item specifica di attori e casi d'uso.
\end{itemize}

Ciascun caso d'uso individua una funzionalit\`a del sistema visibile ad una particolare entit\`a che interagisce col sistema, comunemente chiamata ``attore''.
Un attore pu\`o essere un utente del sistema o un sistema esterno.

I diagrammi dei casi d'uso associano casi d'uso e attori e mostrano la relazione tra i casi d'uso.
La specifica degli attori e dei casi d'uso \`e una descrizione testuale del caso d'uso e ne esplicita i compiti attraverso una descrizione dei passi necessari per realizzare la funzionalit\`a fornita dal caso d'uso.

\section{Attori del sistema}

Di seguito sono illustrati gli attori del sistema.

\section{Specifica dei casi d'uso}

Di seguito \`e illustrata la specifica dei casi d'uso del sistema.

\section{Corrispondenza casi d'uso / requisiti}

Nella matrice seguente \`e illustrata la corrispondenza fra casi d'uso e requisiti funzionali del sistema.
Un segno nella casella in riga $i$, colonna $j$, indica che il caso d'uso in riga $i$ realizza il requisito funzionale in colonna $j$.

\section{Registro modifiche}

\subsection{Elaboration}

\subsubsection{I iterazione}

Prima stesura documento.


%----------------------------------------------------------------------------------------
%	REFERENCE LIST
%----------------------------------------------------------------------------------------

%\nocite{banca_italia}
\printcustombibsmall{}

%----------------------------------------------------------------------------------------
%	FIGURES
%----------------------------------------------------------------------------------------

\end{document}
