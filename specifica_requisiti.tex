
\section{Specifica dei requisiti}

Di seguito \`e illustrata la specifica dettagliata dei requisiti.

\subsection{Requisiti funzionali}

\emph{Da determinare}

\subsection{Requisiti non funzionali}

%TODO aggiungere requisiti di disponibilità e affidabilità

Il sistema di Home Banking deve fornire ai suoi utenti (clienti della banca, dipendenti della banca, agenti di terze parti) la possibilit\`a di visualizzare uno storico dettagliato di ogni operazione (requisito funzionale) effettuata.
Ogni operazione effettuata da un utente (cliente della banca o dipendente della banca) sul sistema deve essere registrata in un log.
Ogni log deve contenere quantomeno le seguenti informazioni:
\begin{itemize}
	\item un'identificativo del tipo di operazione eseguita;
	\item il conto coinvolto nell'operazione;
	\item l'istante dell'operazione;
	\item informazioni riguardanti il terminale da cui \`e stata effettuata l'operazione.
\end{itemize}

\subsubsection{Requisiti di sicurezza del sistema}
 
Il sistema di Home Banking presenta forti requisiti non funzionali di sicurezza. 

\begin{itemize}
	\item L'accesso al sistema di Home Banking da parte di un utente avviene a seguito di autenticazione dello stesso.
		Le credenziali di accesso devono essere trasmesse dal browser dell'utente al sistema di Home Banking utilizzando una connessione sicura.
	\item Le credenziali d'accesso sono per un account sono:
		\begin{itemize}
			\item dati anagrafici del correntista;
			\item numero conto corrente;
			\item password fornita al correntista al momento della registrazione.
		\end{itemize}
	\item Le operazioni effettuabili dall'utente (requisiti funzionali) sono partizionate in operazioni che richiedono autenticazione tramite One Time Password e operazioni che \emph{non} richiedono ulteriore autenticazione.
	Ogni operazione \`e eseguibile solo dopo che l'utente ha effettuato l'accesso al sistema di Home Banking.

	Le seguenti operazioni non richiedono ulteriore autenticazione:
	\begin{itemize}
		\item Visionare saldo contabile, disponibile e liquido.
		\item Visionare uno storico delle transazioni effettuate.
		\item Visionare informazioni riguardo le carte di credito collegate al conto (se presenti).
		\item Effettuare ``operazioni veloci'' impostate attraverso un sistema di configurazione.
	\end{itemize}
	
	Ogni altra operazione richiede autenticazione tramite One Time Password.
	In particolare \`e richiesta autenticazione tramite One Time Password per:
	\begin{itemize}
		\item effettuare transazioni, come:
		\begin{itemize}
			\item bonifici ordinari, bonifici SEPA;
			\item ricariche carte prepatate e schede telefoniche;
			\item pagamento bollette, bollettini, tasse, etc;
		\end{itemize}
		\item configurare operazioni veloci.
	\end{itemize}
\end{itemize}

\subsection{Requisiti di dominio}

\emph{Da determinare}
