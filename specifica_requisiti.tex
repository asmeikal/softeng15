
\section{Specifica dei requisiti}

Di seguito \`e illustrata la specifica dettagliata dei requisiti.

\subsection{Requisiti funzionali}

I requisiti funzionali individuati per il sistema HBS, ossia l'insieme di requisiti che il sistema deve realizzare , sono:
\begin{enumerate}
\item fornire un account HBS ad ogni utente correntista che ne faccia esplicita richiesta e permettergli l'iscrizione mediante apposite procedure;
\item permettere ad un utente registrato l'amministrazione completa e corretta del proprio conto corrente ; per questa ragione sono essenziali le funzionalità che:
\begin{enumerate}
\item permettono all'utente di verificare in tempo reale il suo saldo attuale , il suo saldo contabile e il suo saldo liquido ;
\item permettono all'utente di verificare in tempo reale l'andamento dei titoli azionari che possiede e , nel caso , venderli;
\item permettono all'utente di effettuare operazioni veloci come ricariche telefoniche , bonifici ordinari e pagamento delle bollette;
\end{enumerate}

\item permettere che l'utente possa sfogliare lo storico delle transazioni eseguite, complete delle informazioni riguardanti la data , il destinatario , l'importo previsto e una causale descrittiva delle motivazioni ;
\item permettere che l'utente , in tempo reale , abbia un resoconto delle transazioni che sono in corso da e verso il suo conto;

\item permettere ai dipendenti di una banca , rispettivamente impiegati e manager , di accedere al sistema mediante specifico e personale account ;

\item permettere ai manager di accettare/rifiutare le proposte di bidding per le carte di credito e prestiti fatte dagli utenti;
\item permettere ai manager di inserire tutte le informazioni riguardanti le modalità del bidding offerto dalla banca : soglie minime e massime di offerta, scadenze, eventuali costi, ecc.
\item permettere ai manager di gestire in modo efficiente gli spazi pubblicitari offerti dalle pagine web del sistema, potendone arbitrariamente modificare i contenuti e indirizzando determinate pubblicità ad determinati insiemi di utenti aventi delle caratteristiche desiderate;

\item permettere all'autorità di controllo di accedere ad ogni conto corrente e di recuperare ogni informazione necessaria;
\item permettere all'autorità di controllo di bloccare transazioni in atto e di poter accedere allo storico delle transazioni effettuate;

\end{enumerate}

\subsection{Requisiti non funzionali}

\emph{Da determinare}

\subsubsection{Requisiti di sicurezza del sistema}
 
Le credenziali d'accesso sono per un account sono:
\begin{itemize}
	\item dati anagrafici del correntista;
	\item numero conto corrente;
	\item password scelta dal correntista.
\end{itemize}
	
Una volta che ha effettuato l'accesso,  l'utente deve poter svolgere le seguenti operazioni senza fornire la One Time Password:
\begin{itemize}
	\item Visionare saldo contabile, disponibile e liquido.
	\item Visionare uno storico delle transazioni effettuate.
	\item Visionare informazioni riguardo le carte di credito collegate al conto (se presenti).
	\item Effettuare ``operazioni veloci'' impostate attraverso un sistema di configurazione.
\end{itemize}

Invece è necessario fornire la One Time Password per:
\begin{itemize}
	\item effettuare transazioni, come:
	\begin{itemize}
		\item bonifici ordinari e bonifici SEPA;
		\item ricariche carte prepatate e schede telefoniche;
		\item pagamento bollette, bollettini, tasse, etc;
	\end{itemize}
	\item configurare operazioni veloci;
	\item ogni altra operazione sensibile.
\end{itemize}

Ogni operazione effettuata da un utente sul sistema deve essere registrata in un log.
In particolare, ogni log deve contenere almeno:
\begin{itemize}
	\item l'operazione eseguita;
	\item il conto coinvolto nell'operazione;
		%TODO rivedere codesta parte: codice univoco della transazione al posto del conto?
		% analizzare bene come funziona OBP
	\item l'istante dell'operazione;
	\item informazioni riguardanti il terminale da cui \`e stata effettuata l'operazione.
\end{itemize}

\subsection{Requisiti di dominio}

\emph{Da determinare}
