
\section{Introduzione}

Di seguito sono definite le procedure di test che il sistema di HBS dovr\`a seguire durante e a seguito della fase di \emph{construction} del sistema.

Le procedure di test sono distinte in:
\begin{itemize}
	\item test funzionali, volti alla verifica del funzionamento degli use case previsti dal sistema;
	\item test di sicurezza, volti a controllare la presenza di falle di sicurezza nell'architettura del sistema e alla loro successiva correzione;
	\item test di prestazioni, il cui scopo \`e stabilire in differenti contesti di esecuzione e con differenti metriche le prestazioni del sistema.
\end{itemize}

I test di prestazioni del sistema sono previsti in supporto alla fase di transizione, in cui il software verr\`a venduto agli istituti bancari che ne faranno richiesta, assieme ad un servizio pre-vendita di supporto alla configurazione delle macchine su cui il software dovr\`a girare.
I test di prestazioni del sistema serviranno come indicatori per individuare la corretta configurazione delle macchine, rispondendo alle richieste del cliente.

\subsection{Struttura Test Funzionali per HBS}

Il sistema di HBS si basa sull'interfaccia OBP per la comunicazione con il back-end della banca.
I test funzionali vedono coinvolti tre elementi:
\begin{itemize}
	\item il sistema HBS;
	\item l'interfaccia OBP;
	\item il back-end della banca.
\end{itemize}
Per verificare il funzionamento degli use-case principali (che coinvolgono quindi la comunicazione con il back-end tramite OBP), HBS verr\`a testato su un sistema bancario d'esempio, ossia un back-end che permetta di testare gli eventi rilevanti collegato tramite OBP al sistema HBS.

Il back-end su cui HBS verr\`a testato non ha requisiti di consistenza.
Il requisito fondamentale \`e che faciliti la copertura di tutti gli eventi che possono avvenire durante la comunicazione (mediata da OBP) fra HBS e back-end.
Gli eventi rilevanti identificati sono:
\begin{enumerate}
	\item inserimento di un'operazione (use case \iducDISPAG), distinguendo fra i possibili esiti:
		\begin{enumerate}
			\item operazione accettata;
			\item operazione rifiutata (ad es.\ causa mancanza fondi).
		\end{enumerate}
	\item richiesta storico operazioni (use case \iducVERSTOR);
	\item richiesta saldo conto (use case \iducVERSAL).
\end{enumerate}

\subsection{Struttura Test di Prestazioni}

I test di prestazioni del sistema sono volti ad ottenere una misura indicativa del tempo impiegato da HBS a rispondere alle richieste dei clienti.
A differenza dei test funzionali, i test di prestazioni richiedono un back-end dalla struttura pi\`u elaborata, che permetta di osservare il comportamento del sistema in scenari simili agli scenari che HBS dovr\`a realmente affrontare.

