\section{Specifica business plan}

\subsection{Tipologie di contratti e rapporto col cliente}
La nostra società si propone di vendere il software del sistema HBS a banche e/o società pubbliche di investimento che necessitino di un sistema di home banking efficiente , customizzabile e \emph{a norma di legge}.

Per ottenere il massimo profitto , essendo questa un nuovo ambito di forte interesse economico , anche per via della stretta vicinanza all'ambiente della finanza, la società punta a :
\begin{enumerate}
\item pubblicizzare ampiamente il prodotto per imporsi in questa nuova area di mercato;
\item rendere il sistema il più customizzabile possibile , affinchè si adatti pienamente alle esigenze  la banca che lo adotti;
\item assistere il cliente anche nella fase successiva all'acquisto del prodotto , ad esempio svolgendo regolare manutenzione , o aggiornando periodicamente il software.
\end{enumerate} 

Ci sono due tipi di contratto per il rilascio del \emph{sistema hbs} :
\begin{enumerate}
\item vendita di componenti hardware prestabilite dalla nostra società , noleggio del software , installazione , regolare manutenzione  e aggiornamenti periodici del software;
\item noleggio del software ,installazione, regolare manutenzione e aggiornamenti periodici del software.
\end{enumerate}

La vendita del prodotto HBS prevede l'installazione del sistema da parte di personale specializzato della nostra azienda su dispositivi e secondo alcune specifiche indicazioni dell'acquirente. In assenza del software e/o delle macchine necessarie all'installazione o al corretto funzionamento del sistema HBS la nostra societa si impegna a fornire l’hardware necessario predisposto all’installazione a costo aggiuntivo (contratto del primo tipo).
Un contratto del secondo tipo può essere concordato esclusivamente previo controllo ispettivo dei componenti hardware su cui l'acquirente intende installare il sistema : il controllo è mirato a certificare che tali componenti possiedano i requisiti necessari a far funzionare correttamente il software noleggiato.

La nostra società garantisce le soglie di uptime e downtime solo per i contratti di primo tipo , esimendosi da ogni responsabilità riguardante prestazioni hardware inferiori a quelle pattuite con l'acquirente ( ogni responsabilità ricade sul venditore cui la nostra società si rivolge per l'acquisto dell'hardware).

L'acquirente , che acquisisce i diritti di compiere le operazioni dovute sul sistema firmando il \emph{contratto di noleggio} , possiede su nostra concessione il software noleggiato , sul quale perde ogni diritto di (usabilità-usufrutto) al momento della rescissione del contratto.



\subsection{Obiettivi di guadagno}
La nostra azienda punta a guadagnare , nel breve termine , con la stipulazione del contratto di noleggio,  nel medio-lungo termine , con la manutenzione e periodici aggiornamenti software del sistema . 
Per queste ragioni , il pagamento viene effettuato dall'acquirente :
\begin{enumerate}
\item alla firma di una delle due tipologie di contratto , con particolare riferimento al solo contratto di noleggio per il software Hbs;
\item ad ogni intervento manutentivo di dipendenti della nostra azienda;
\item all'acquisto (sempre sottoforma di contratto di noleggio) di patch e aggiornamenti del software rilasciati dalla nostra azienda.
\end{enumerate}

\subsection{ Peculiarità del prodotto e vendita }
Il software Hbs ha due principali componenti:
\begin{enumerate}
\item il sistema Hbs in sè , comprendente tutte le pagine del sito web , tutte le funzionalità di gestione dello stesso lato banca e (per i conti correnti) lato utente , il sistema di controllo per le autorità competenti e , in generale , tutti i meccanismi e i protocolli di sicurezza adottati ;
\item programmi di creazione e manipolazione di contenuti delle pagine del sito web .
\end{enumerate}
Queste due componenti sono divise tra loro e hanno compiti diversi : mentre la prima si occupa della gestione vera e propria di un sistema di home banking , la seconda serve principalmente a dipendenti e manager della banca per modificare form e contenuti da aggiungere alle pagine del sito web (ad esempio , aggiungere una nuova \emph{operazione veloce }, non contenuta tra quelle del set base).
Poichè si vuole rendere indipendente il sistema Hbs dalla specifica piattaforma per aumentarne la compatibilità e massimizzare i profitti, il reale prodotto messo in vendita è un' \emph{istanza del sistema Hbs} , ossia una copia  del software originale in nostro possesso  "customizzata" secondo le esigenze del cliente specifico (la customizzazione riguarda l'aspetto grafico delle pagine web rivolte verso i correntisti o verso gli addetti della banca stessa) e inserita in un'immagine (per le macchine virtuali) o in un container (per i docker) . Dell'istanza in vendita faranno parte le componenti Hbs del secondo tipo.
 



\subsection{Prerequisiti hardware}
Affinchè il sistema sia installabile , i dispositivi target dovranno soddisfare determinati requisiti , che saranno indicati in una \emph{lista di prerequisiti hardware} e che costituiscono l'insieme minimo di risorse hardware che un calcolatore deve possedere affinchè il software HBS produca prestazioni ottimali.
L'unica risorsa software richiesta (unico prerequisito del prodotto in vendita) è la presenza , sul dispositivo target , di una qualsivoglia piattaforma di virtualizzazione di OS (macchine virtuali o docker , a scelta del cliente).
 









