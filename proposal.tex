\documentclass[]{softeng}

\title{Proposta di progetto Home Banking}
\date{\today}

\begin{document}

\maketitle

\section{Visione del sistema}

Si vuole realizzare un sistema di Home Banking rivolto a persone fisiche (retail banking).
Considerata la necessit\`a crescente di aumentare la trasparenza bancaria e i controlli fiscali automatizzati, il governo ha varato\footnote{Quanto segue non rispecchia interamente la realt\`a.} un pacchetto di leggi che impone alle societ\`a di credito italiane di adeguarsi allo standard OBP\footnote{Open Bank Project: \url{https://openbankproject.com/}} gi\`a in uso in Germania e nel Regno Unito\footnote{\url{http://opensource.com/business/15/4/open-standard-api-banking}}.
In questo contesto di transizione, la nostra societ\`a si propone di progettare un'applicazione che faccia da front-end per una banca che abbia adottato lo standard OBP.

Lo standard OBP \`e fondamentalmente un metodo standardizzato di accesso ai dati contenuti nel back-end della banca.
Nel momento in cui si adegua allo standard, ogni banca mantiene inalterate la struttura del suo database e dei sistemi preesistenti atti alla gestione delle operazioni bancarie.
Il software di OBP agisce quindi da \emph{wrapper} del sistema preesistente, mascherando completamente e rendendo inaccessibili le interfacce del backend, ed esponendo all'esterno unicamente le interfacce definite dallo standard.

\graphicspath{{Images/}}
\begin{figure}[tp]
	\centering
	\includegraphics[width=\textwidth]{open_bank_project_architecture}
	\caption{Architettura del software di Open Bank Project. \url{https://github.com/OpenBankProject/OBP-API/wiki/Open-Bank-Project-Architecture}}
	\label{fig:open_bank_project_architecture}
\end{figure}
% TODO inserire diagramma architetturale di OBP

La nostra applicazione permetter\`a ai clienti di effettuare le normali operazioni bancarie da casa, accedendo al sito della banca tramite web browser.

Il nostro sistema dovr\`a:
\begin{enumerate}
	\item Permettere la registrazione online di nuovi clienti.
		La registrazione deve essere ultimata presso una filiale della banca.
	\item Utilizzare lo standard TOTP\footnote{CIAO}.
	\item Permettere all'utente di effettuare le seguenti operzioni:
		\begin{enumerate}
			\item visualizzare il saldo (contabile, disponibile e liquido)
			\item visualizzare lo storico delle transazioni effettuate
			\item gestire le proprie carte di credito e carte prepagate
			\item disporre pagamenti di vario tipo (bonifici ordinari, bonifici SEPA, ricariche prepagate, ricariche telefoniche, pagamenti bollettini, etc)
			\item gestire portafogli fondi, azioni, obbligazioni e titoli statali % TODO CHECK SYNTAX
		\end{enumerate}
	\item Permette di effettuare l'audit di sicurezza agli addetti della banca.
	\item Notificare all'utente (tramite SMS e/o email) determinati eventi.
\end{enumerate}

Simuleremo la realizzazione del progetto seguendo lo Unified Process, producendo la documentazione richiesta da ogni iterazione e flusso di lavoro.
L'implementazione del sistema non fa parte di questo progetto.

\section{Documentazione}

La documentazione che produrremo comprender\`a:
\begin{enumerate}
	\item Studio di fattibilit\`a e analisi del contesto.
	\item Specifica dei requisiti e modello di business.
	\item Documento di analisi dei rischi.
	\item Bozza di contratto per il cliente (basata sulla specifica dei requisiti prodotta).
	\item Analisi del sistema % TODO espandere
	\item Analisi dei costi e pianificazione temporale.
	\item Piano dei test.
\end{enumerate}
L'ordinamento dei documenti non \`e vincolante, ma \`e stato opportunamente scelto per i seguenti motivi:
\begin{itemize}
	\item L'analisi dei rischi precede la bozza di contratto perch\'e il sistema che intendiamo sviluppare ha forti requisiti di sicurezza, ed \`e opportuno iniziare l'analisi dei rischi nelle fasi iniziali del processo software.
\end{itemize}

\end{document}

