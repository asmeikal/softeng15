\documentclass[draft]{softeng}

\title{Proposta di progetto Home Banking}
\date{\today}

\begin{document}

\maketitle

\section{Proposta di progetto} 

Si vuole realizzare un sistema di Home Banking.

Per rafforzare i requisiti non funzionali (come i requisiti di sicurezza) si sceglie di rivolgere il sistema di Home Banking a persone fisiche (retail banking).

Alcune delle funzionalit\`a che il sistema pu\`o offrire sono:
\begin{enumerate}
	\item L'utente si deve poter pre-registrare online fornendo i suoi dati anagrafici, e:
		\begin{enumerate}
			\item se l'utente ha gi\`a un conto aperto con la banca, pu\`o fornire il suo numero di conto;
			\item se l'utente non ha gi\`a un conto con la banca, pu\`o scegliere diverse soluzioni con agevolazioni differenti e pre-compilare i moduli necessari all'apertura del conto.
				% TODO: cambiare "se l'utente ha gi\`a un conto con la banca" in "se l'utente desidera aprire un nuovo conto"
		\end{enumerate}
		Un utente pre-registrato pu\`o recarsi presso una filiale della banca e ultimare la registrazione presentando il documento d'identit\`a.
		La banca gli fornir\`a le informazioni per l'accesso.
		Ai nuovi clienti viene fornito su richiesta un bancomat.
		% TODO: rimuovere, non \`e di interesse per l'applicazione, facciamo home banking non ATM

		Un utente non pre-registrato pu\`o effettuare la procedura completa di registrazione presso una filiale, fornendo le stesse informazioni richieste agli utenti che si pre-registrano online.
	\item Le informazioni per l'accesso comprendono:
		\begin{enumerate}
			\item Il numero di conto;
			\item La password del conto;
			\item Un dispositivo TOTP\footnote{Time-Based One Time Password - \href{https://tools.ietf.org/html/rfc6238}{RFC 6238}.} per effettuare operazioni sensibili.
		\end{enumerate}
	\item Un utente registrato pu\`o effettuare il login al sistema di Home Banking fornendo le informazioni di accesso (requisito TOT), in particolare deve inserire numero di conto e password del conto.
	\item Un utente che ha effettuato l'accesso al sistema di Home Banking deve poter svolgere le seguenti operazioni senza fornire la One Time Password:
		\begin{enumerate}
			\item Visionare saldo contabile, disponibile e liquido.
			\item Visionare uno storico delle transazioni effettuate.
			\item Visionare informazioni riguardo le carte di credito collegate al conto (se presenti).
			\item Effettuare ``operazioni veloci'' impostate attraverso un sistema di configurazione (requisito TOT).
		\end{enumerate}
	\item Un utente che ha effettuato l'accesso al sistema di Home Banking deve poter effettuare le seguenti operazioni fornendo ogni volta la One Time Password:
		\begin{enumerate}
			\item Effettuare transazioni, come:
				\begin{enumerate}
					\item bonifici ordinari e bonifici SEPA;
					\item ricariche carte prepatate e schede telefoniche;
					\item pagamento bollette, bollettini, tasse, etc;
				\end{enumerate}
			\item Configurare operazioni veloci.
			\item Se ci viene in mente altro ce lo mettiamo.
		\end{enumerate}
	\item Ogni operazione deve essere registrata in un log.
		Devono essere mantenute le seguenti informazioni:
		\begin{enumerate}
			\item l'operazione eseguita;
			\item il conto coinvolto nell'operazione;
				% TODO rivedere codesta parte: codice univoco della transazione al posto del conto?
				% analizzare bene come funziona OBP
			\item l'istante dell'operazione;
			\item informazioni riguardanti il terminale da cui \`e stata effettuata l'operazione.
		\end{enumerate}
	\item Il sistema deve poter essere configurabile dall'utente per inviare notifiche via SMS e/o via email a seguito di ogni evento stabilito dall'utente.
		La banca pu\`o associare una tariffa per abilitare le notifiche su specifici insiemi di eventi.
		Gli eventi notificabili possono includere:
		\begin{enumerate}
			\item Accesso al sistema;
			\item Pagamento o prelievo dal conto;
			\item Pagamento ricevuto.
		\end{enumerate}
\end{enumerate}

\end{document}

