\subsection{\code{UC\_1} - Verifica Storico }
\label{sec:use-case:VERSTOR}

\begin{ptable}{2}
\ptitle{Titolo}
\pcell{1}{
	Verifica Storico (\code{UC\_VERSTOR\_1})
}
\pline
\ptitle{Descrizione use case}
\pcell{1}{
		Un cliente di HBS deve poter visualizzare lo storico delle operazioni effettuate dal suo conto corrente in una finestra temporale da lui specificata.
}
\pline
\ptitle{Attori}
\pcell{1}{
		Cliente di HBS.
}
\pline
\ptitle{Origine}
\pcell{1}{
		Requisiti funzionali (requisito di sistema \idVERSTOR).
}
\pline
\ptitle{Pre-condizioni}
\pcell{1}{
		Nessua.
}
\pline
\ptitle{Flusso}
\pcell{1}{
		\begin{enumerate} \item Il cliente di HBS naviga alla pagina di gestione dello storico dopo aver effettuato il login al sistema; \item il cliente di HBS seleziona un conto corrente fra quelli che possiede presso l'istituto, e specifica un periodo di riferimento per cui desidera ottenere l'elenco delle operazioni effettuate; \item il cliente di HBS invia i parametri specificati al sistema; \item il sistema recupera le informazioni richieste e le mostra al cliente. \end{enumerate}
}
\pline
\ptitle{Post-condizioni}
\pcell{1}{
		Nessuna.
}
\pline
\ptitle{Side effects}
\pcell{1}{
		Nessuno.
}
\end{ptable}

\subsection{\code{UC\_2} - Visualizzazione Bidding }
\label{sec:use-case:BIDVIS}

\begin{ptable}{2}
\ptitle{Titolo}
\pcell{1}{
	Visualizzazione Bidding (\code{UC\_BIDVIS\_2})
}
\pline
\ptitle{Descrizione use case}
\pcell{1}{
		I manager della banca possono ottenere delle visualizzazioni delle regole di bidding realizzate. Una visualizzazione fornisce una rappresentaziona grafica intuitiva dell'insieme di bid approvati automaticamente, soggetti ad approvazione da parte di un manager, e respinti automaticamente.
}
\pline
\ptitle{Attori}
\pcell{1}{
		Manager della banca
}
\pline
\ptitle{Origine}
\pcell{1}{
		Requisiti funzionali e requisiti di usabilit\`a.
}
\pline
\ptitle{Pre-condizioni}
\pcell{1}{
		Almeno una regola di bidding \`e stata definita.
}
\pline
\ptitle{Flusso}
\pcell{1}{
		\begin{enumerate} \item Il manager seleziona la sezione di visualizzazione delle regole di bidding definite;
\item il manager seleziona una regola di bidding;
\item il sistema di HBS realizza una visualizzazione della regola di bidding;
\item la visualizzazione prodotta viene mostrata al manager nel client. \end{enumerate}
}
\pline
\ptitle{Post-condizioni}
\pcell{1}{
		Nessuna
}
\pline
\ptitle{Side effects}
\pcell{1}{
		Nessuno
}
\end{ptable}

\subsection{\code{UC\_3} - Disposizione di Pagamento }
\label{sec:use-case:DISPAG}

\begin{ptable}{2}
\ptitle{Titolo}
\pcell{1}{
	Disposizione di Pagamento (\code{UC\_DISPAG\_3})
}
\pline
\ptitle{Descrizione use case}
\pcell{1}{
		Un cliente di HBS deve poter effettuare disposizioni di pagamento generiche, quali bonifici e bonifici SEPA.
Le disposizioni di pagamento necessitano di ulteriore autenticazione tramite OTP.
}
\pline
\ptitle{Attori}
\pcell{1}{
		Cliente di HBS.
}
\pline
\ptitle{Origine}
\pcell{1}{
		Requisiti funzionali (requisito di sistema \idDISPAG, sez.~\ref{req:sec:sistema:funzionali:DISPAG} del documento dei requisiti).
}
\pline
\ptitle{Pre-condizioni}
\pcell{1}{
		Il cliente di HBS ha i permessi per effettuare operazioni dal conto corrente selezionato.
}
\pline
\ptitle{Flusso}
\pcell{1}{
		\begin{enumerate} \item Il cliente seleziona la pagina relativa all'invio di operazioni;
\item il cliente di HBS seleziona il tipo di operazione da effettuare (bonifico, bonifico SEPA, etc);
\item il sistema presenta al cliente il form da compilare per effettuare le operazioni;
\item il cliente compila i campi necessari per effettuare l'operazione (somma di denaro, data di valuta dell'operazione, conto corrente del destinatario, etc);
\item il cliente invia il form compilato al sistema di HBS;
\item il sistema di HBS richiede OTP al cliente di HBS tramite apposito form;
\item \label{itm:dispag:last} il cliente inserisce e invia la OTP al sistema;
\item \label{itm:conferma} il sistema conferma la presa in carico dell'operazione da parte del back-end della banca. \end{enumerate}
}
\pline
\ptitle{Post-condizioni}
\pcell{1}{
		L'operazione indicata dall'utente \`e stata presa in carico con successo dal back-end della banca.
}
\pline
\ptitle{Side effects}
\pcell{1}{
		Nessuno.
}
\pline
\ptitle{Flusso alternativo}
\pcell{1}{
		L'utente pu\`o interrompere l'operazione in qualunque momento prima del punto~\ref{itm:dispag:last} senza che questa abbia conseguenze.
Nel caso la OTP ricevuta dal sistema al punto~\ref{itm:conferma} non sia corretta, o nel caso in cui il back-end rifiuti l'operazione, lo use case procede con il flusso alternativo.
}
\pline
\ptitle{Post-condizioni alternative}
\pcell{1}{
		Nessuna.
}
\pline
\ptitle{Side effects alternativi}
\pcell{1}{
		Nessuno.
}
\end{ptable}

\subsection{\code{UC\_4} - Creazione Operazione Veloce }
\label{sec:use-case:CROPVEL}

\begin{ptable}{2}
\ptitle{Titolo}
\pcell{1}{
	Creazione Operazione Veloce (\code{UC\_CROPVEL\_4})
}
\pline
\ptitle{Descrizione use case}
\pcell{1}{
		I dipendenti della banca devono poter creare delle operazioni veloci specificando il metodo di traduzione dei parametri delle operazioni veloci nei parametri dell'operazione tradizionale corrispondente.
}
\pline
\ptitle{Attori}
\pcell{1}{
		Dipendente della banca.
}
\pline
\ptitle{Origine}
\pcell{1}{
		Requisiti funzionali (requisito di sistema \idCROPVEL, sez.~\ref{req:sec:sistema:funzionali:CROPVEL} del documento dei requisiti).
}
\pline
\ptitle{Pre-condizioni}
\pcell{1}{
		Le operazioni veloci sono state abilitate dal manager della banca.
}
\pline
\ptitle{Flusso}
\pcell{1}{
		\begin{enumerate} \item Il dipendente seleziona la pagina di creazione delle operazioni veloce;
\item il dipendente specifica il nome dell'operazione veloce;
\item il dipendente seleziona il tipo di operazione tradizionale in cui l'operazione veloce deve essere convertita;
\item per ciascun parametro dell'operazione tradizionale il dipendente specifica un parametro dell'operazione veloce con relativo dominio e meccanismo (algoritmo) di traduzione, o una costante;
\item \label{itm:CROPVEL:last} il dipendente salva l'operazione veloce. \end{enumerate}
}
\pline
\ptitle{Post-condizioni}
\pcell{1}{
		Nessuna.
}
\pline
\ptitle{Side effects}
\pcell{1}{
		L'operazione veloce creata dal dipendente della banca \`e stata memorizzata nel sistema di HBS.
}
\pline
\ptitle{Flusso alternativo}
\pcell{1}{
		Prima del punto~\ref{itm:CROPVEL:last} il dipendente pu\`o annullare l'operazione senza provocare cambiamenti al sistema.
}
\pline
\ptitle{Post-condizioni alternative}
\pcell{1}{
		Nessuna.
}
\pline
\ptitle{Side effects alternativi}
\pcell{1}{
		Nessuno.
}
\end{ptable}

\subsection{\code{UC\_5} - Accesso Clienti }
\label{sec:use-case:CLIACC}

\begin{ptable}{2}
\ptitle{Titolo}
\pcell{1}{
	Accesso Clienti (\code{UC\_CLIACC\_5})
}
\pline
\ptitle{Descrizione use case}
\pcell{1}{
		Un cliente di HBS che sia stato approvato dai dipendenti della banca deve poter accedere al sistema tramite web browser.
}
\pline
\ptitle{Attori}
\pcell{1}{
		Cliente di HBS.
}
\pline
\ptitle{Origine}
\pcell{1}{
		Requisiti funzionali (requisito di sistema \idCLIACC).
}
\pline
\ptitle{Pre-condizioni}
\pcell{1}{
		Il cliente di HBS \`e stato approvato dai dipendenti della banca.
}
\pline
\ptitle{Flusso}
\pcell{1}{
		\begin{enumerate} \item L'utente naviga fino alla pagina di accesso a HBS della sua banca utilizzando un web browser; \item la pagina di accesso richiede identificativo utente e password di accesso; \item \label{itm:CLIACC:invio-credenziali} l'utente inserisce identificativo e password e preme il pulsante di invio; \item identificativo e password sono inviati al sistema di HBS tramite connessione sicura; \item \label{itm:CLIACC:controllo-credenziali} identificativo e password vengono confrontati con i valori memorizzati nel database di HBS; \item il sistema di HBS redireziona il browser del cliente alla pagina principale di HBS. \end{enumerate}
}
\pline
\ptitle{Post-condizioni}
\pcell{1}{
		Nessuna.
}
\pline
\ptitle{Side effects}
\pcell{1}{
		Nessuno.
}
\pline
\ptitle{Flusso alternativo}
\pcell{1}{
		In qualsiasi momento prima del punto \ref{itm:CLIACC:invio-credenziali} l'utente pu\`o interrompere l'operazione senza inviare alcuna informazione alla banca. Qualora le informazioni ricevute da HBS al punto \ref{itm:CLIACC:controllo-credenziali} non corrispondano con le informazioni memorizzate nel database di HBS, il sistema redireziona l'utente ad una pagina di errore.
}
\pline
\ptitle{Post-condizioni alternative}
\pcell{1}{
		Nessuna.
}
\pline
\ptitle{Side effects alternativi}
\pcell{1}{
		Nel caso in cui il testo al punto \ref{itm:CLIACC:controllo-credenziali} fallisca, il tentativo di accesso fallito viene memorizzato nei log di sicurezza.
}
\end{ptable}

\subsection{\code{UC\_6} - Bidding Utente }
\label{sec:use-case:USRBID}

\begin{ptable}{2}
\ptitle{Titolo}
\pcell{1}{
	Bidding Utente (\code{UC\_USRBID\_6})
}
\pline
\ptitle{Descrizione use case}
\pcell{1}{
		Un utente di HBS pu\`o effettuare bid per conti correnti, carte di credito o prestiti.
}
\pline
\ptitle{Attori}
\pcell{1}{
		Utente di HBS
}
\pline
\ptitle{Origine}
\pcell{1}{
		Requisiti funzionali (requisito utente~\ref{req:itm:utente:funzionali:bidding:utente}, sez.~\ref{req:sec:utente:funzionali}, e requisito di sistema \idUSRBID).
}
\pline
\ptitle{Pre-condizioni}
\pcell{1}{
		Il sistema di bidding \`e stato attivato dalla banca.
}
\pline
\ptitle{Flusso}
\pcell{1}{
		\begin{enumerate} \item L'utente seleziona la pagina di bidding;
\item l'utente seleziona l'oggetto per cui desidera fare bid fra quelli per cui l'istituto permette operazioni di bid;
\item il sistema presenta la pagina per l'inserimento dei parametri di bid;
\item l'utente specifica i parametri del proprio bid;
\item l'utente invia il bid al sistema di HBS;
\item l'utente riceve un responso dal sistema di HBS riguardo l'approvazione, il rifiuto o la presa in carico del bid da parte del sistema stesso. \end{enumerate}
}
\pline
\ptitle{Post-condizioni}
\pcell{1}{
		Nessuna.
}
\pline
\ptitle{Side effects}
\pcell{1}{
		Se la proposta di bid \`e stata rifiutata, nessuno. Se la proposta di bid \`e stata accettata, il bid \`e stato memorizzato nel sistema come bid approvato automaticamente, e inviato a un dipendente della banca per la realizzazione. Se la proposta di bid \`e stata presa in carico per il controllo da parte di un manager della banca, la proposta di bid \`e stata inserita nel sistema come proposta non ancora approvata, ed \`e stata inviata al manager della banca.
}
\pline
\ptitle{Flusso alternativo}
\pcell{1}{
		In qualsiasi punto del workflow l'utente pu\`o interrompere l'inserimento della proposta di bid.
}
\pline
\ptitle{Post-condizioni alternative}
\pcell{1}{
		Nessuna.
}
\pline
\ptitle{Side effects alternativi}
\pcell{1}{
		Nessuno.
}
\end{ptable}

\subsection{\code{UC\_7} - Creazione Bidding }
\label{sec:use-case:CREABID}

\begin{ptable}{2}
\ptitle{Titolo}
\pcell{1}{
	Creazione Bidding (\code{UC\_CREABID\_7})
}
\pline
\ptitle{Descrizione use case}
\pcell{1}{
		Un manager della banca pu\`o creare una regola di bidding.
}
\pline
\ptitle{Attori}
\pcell{1}{
		Manager della banca.
}
\pline
\ptitle{Origine}
\pcell{1}{
		Requisiti funzionali (requisito di sistema \idCREABID).
}
\pline
\ptitle{Pre-condizioni}
\pcell{1}{
		Nessuna.
}
\pline
\ptitle{Flusso}
\pcell{1}{
		\begin{enumerate} \item Il manager seleziona la pagina di creazione dei bid;
\item il manager seleziona una tipologia di bidding (carte di credito, conto corrente, prestito) per cui desidera impostare i parametri di scelta;
\item il sistema presenta la schermata di scelta dei parametri di bid;
\item \label{itm:uc:CREABID:parametro} il manager seleziona un parametro di scelta;
\item il manager imposta i range di approvazione automatica, approvazione condizionata, e rifiuto automatico sul parametro scelto;
\item opzionalmente, il manager torna al punto \ref{itm:uc:CREABID:parametro};
\item il manager inserisce la regola di bidding creata. \end{enumerate}
}
\pline
\ptitle{Post-condizioni}
\pcell{1}{
		Nessuna.
}
\pline
\ptitle{Side effects}
\pcell{1}{
		La regola di bidding realizzata \`e stata aggiunta al sistema di HBS.
}
\pline
\ptitle{Flusso alternativo}
\pcell{1}{
		In qualsiasi punto del workflow il manager pu\`o annullare l'operazione.
}
\pline
\ptitle{Post-condizioni alternative}
\pcell{1}{
		Nessuna.
}
\pline
\ptitle{Side effects alternativi}
\pcell{1}{
		Nessuno.
}
\end{ptable}

\subsection{\code{UC\_8} - Iscrizione Correntista }
\label{sec:use-case:ISCRCORR}

\begin{ptable}{2}
\ptitle{Titolo}
\pcell{1}{
	Iscrizione Correntista (\code{UC\_ISCRCORR\_8})
}
\pline
\ptitle{Descrizione use case}
\pcell{1}{
		Un cliente della banca non registrato ad HBS deve poter effettuare la pre-registrazione in remoto.
}
\pline
\ptitle{Attori}
\pcell{1}{
		Cliente della banca.
}
\pline
\ptitle{Origine}
\pcell{1}{
		Requisiti funzionali (requisito di sistema \idISCRCORR, sez.~\ref{req:sec:sistema:funzionali:ISCRCORR} del documento dei requisiti).
}
\pline
\ptitle{Pre-condizioni}
\pcell{1}{
		Il cliente della banca non \`e registrato a HBS, ossia non \`e un cliente di HBS.
}
\pline
\ptitle{Flusso}
\pcell{1}{
		\begin{enumerate} \item Il cliente della banca visita la pagina per l'iscrizione;
\item il cliente della banca inserisce il proprio codice fiscale e il proprio numero di conto corrente nella pagina di iscrizione;
\item il cliente invia le informazioni al sistema di HBS;
\item il sistema prende in carico la richiesta di iscrizione. \end{enumerate}
}
\pline
\ptitle{Post-condizioni}
\pcell{1}{
		Nessuna.
}
\pline
\ptitle{Side effects}
\pcell{1}{
		La richiesta di iscrizione \`e stata presa in carico dal sistema di HBS.
}
\end{ptable}

\subsection{\code{UC\_9} - Approvazione cliente pre-registrato }
\label{sec:use-case:APPRCORR}

\begin{ptable}{2}
\ptitle{Titolo}
\pcell{1}{
	Approvazione cliente pre-registrato (\code{UC\_APPRCORR\_9})
}
\pline
\ptitle{Descrizione use case}
\pcell{1}{
		Un cliente di HBS che abbia effettuato la pre-registrazione online e che non sia stato ancora confermato da un dipendente della banca pu\`o richiedere l'approvazione del proprio account presso una filiale della sua banca.
}
\pline
\ptitle{Attori}
\pcell{1}{
		Dipendente della banca, cliente di HBS.
}
\pline
\ptitle{Origine}
\pcell{1}{
		Requisiti funzionali (requisito di sistema \idAPPRCORR).
}
\pline
\ptitle{Pre-condizioni}
\pcell{1}{
		Il cliente di HBS si \`e pre-registrato ma non ha ancora ultimato l'iscrizione.
}
\pline
\ptitle{Flusso}
\pcell{1}{
		\begin{enumerate} \item Il cliente di HBS si reca presso una filiale della sua banca per ultimare la registrazione; \item il cliente di HBS fornisce ad un dipendente della banca le informazioni richieste dalla testa per verificarne l'identit\`a; \item il dipendente della banca coinvolto nell'interazione raggiunge la pagina di approvazione dei clienti tramite il software di HBS; \item \label{itm:APPROCORR:approvazione} il dipendente della banca approva l'iscrizione del cliente di HBS; \item il sistema genera un identificativo e una password per il cliente di HBS; \item identificativo e password vengono fornite in busta chiusa al cliente di HBS. \end{enumerate}
}
\pline
\ptitle{Post-condizioni}
\pcell{1}{
		Il cliente di HBS risulta regolarmente iscritto.
}
\pline
\ptitle{Side effects}
\pcell{1}{
		Il cliente di HBS non appartiene pi\`u all'insieme dei clienti non ancora registrati.
}
\pline
\ptitle{Flusso alternativo}
\pcell{1}{
		Al punto \ref{itm:APPROCORR:approvazione} il dipendente della banca pu\`o non approvare l'iscrizione del cliente qualora le politiche di approvazione della banca non siano verificate.
}
\pline
\ptitle{Post-condizioni alternative}
\pcell{1}{
		Nessuna.
}
\pline
\ptitle{Side effects alternativi}
\pcell{1}{
		Nessuno.
}
\end{ptable}

\subsection{\code{UC\_10} - Disposizione Operazione Veloce }
\label{sec:use-case:DISOPVEL}

\begin{ptable}{2}
\ptitle{Titolo}
\pcell{1}{
	Disposizione Operazione Veloce (\code{UC\_DISOPVEL\_10})
}
\pline
\ptitle{Descrizione use case}
\pcell{1}{
		Un cliente di HBS deve poter effettuare le operazioni veloci configurate dalla banca.
}
\pline
\ptitle{Attori}
\pcell{1}{
		Cliente di HBS.
}
\pline
\ptitle{Origine}
\pcell{1}{
		Requisiti funzionali (requisito di sistema \idDISOPVEL, sez.~\ref{req:sec:sistema:funzionali:DISOPVEL} del documento dei requisiti).
}
\pline
\ptitle{Pre-condizioni}
\pcell{1}{
		Almeno una tipologia di operazione veloce \`e stata configurata dai dipendenti della banca.
}
\pline
\ptitle{Flusso}
\pcell{1}{
		\begin{enumerate} \item L'utente visita la pagina per la disposizione di operazioni veloci;
\item l'utente seleziona la tipologia di operazione veloce che desidera effettuare fra quelle disponibili;
\item il sistema presenta all'utente la pagina per l'inserimento dei parametri dell'operazione veloce;
\item l'utente inserisce i parametri richiesti dall'operazione veloce;
\item l'utente invia l'operazione veloce al sistema di HBS;
\item \label{itm:DISOPVEL:validazione} il sistema di HBS valida e traduce l'operazione veloce in operazione ordinaria;
\item il sistema di HBS richiede la OTP al cliente di HBS tramite l'apposita pagina;
\item \label{itm:DISOPVEL:conferma-cliente} il cliente di HBS invia al sistema la OTP;
\item \label{itm:DISOPVEL:otp} il sistema di HBS valida la OTP;
\item \label{itm:DISOPVEL:back-end} il sistema di HBS invia l'operazione ordinaria risultante dalla traduzione dell'operazione veloce al sistema di OBP;
\item il sistema di HBS restituisce all'utente conferma dell'avenuta presa in carico da parte del back-end della banca dell'operazione ordinaria corrispondente all'operazione veloce richiesta. \end{enumerate}
}
\pline
\ptitle{Post-condizioni}
\pcell{1}{
		Nessuna.
}
\pline
\ptitle{Side effects}
\pcell{1}{
		L'operazione ordinaria corrispondente all'operazione veloce richiesta dall'utente \`e stata presa in carico correttamente dal back-end della banca.
}
\pline
\ptitle{Flusso alternativo}
\pcell{1}{
		Il cliente di HBS pu\`o interrompere l'operazione in qualunque momento prima del punto~\ref{itm:DISOPVEL:conferma-cliente} senza produrre risultati. Il sistema pu\`o restituire un errore ai punti~\ref{itm:DISOPVEL:validazione}, \ref{itm:DISOPVEL:otp} e \ref{itm:DISOPVEL:back-end}, nel caso in cui, rispettivamente, i parametri dell'operazione veloce non siano validi, la OTP non sia valida, e il back-end rifiuti l'operazione ordinaria risultante. In questo caso non viene prodotto alcun risultato e l'utente visualizza una pagina di errore.
}
\pline
\ptitle{Post-condizioni alternative}
\pcell{1}{
		Nessuna.
}
\pline
\ptitle{Side effects alternativi}
\pcell{1}{
		Nessuno.
}
\end{ptable}

\subsection{\code{UC\_11} - Verifica Saldo }
\label{sec:use-case:VERSAL}

\begin{ptable}{2}
\ptitle{Titolo}
\pcell{1}{
	Verifica Saldo (\code{UC\_VERSAL\_11})
}
\pline
\ptitle{Descrizione use case}
\pcell{1}{
		Un cliente di HBS deve poter verificare il saldo contabile, attuale e liquido dei suoi conti correnti.
}
\pline
\ptitle{Attori}
\pcell{1}{
		Cliente di HBS.
}
\pline
\ptitle{Origine}
\pcell{1}{
		Requisiti funzionali (requisito di sistema \idVERSAL).
}
\pline
\ptitle{Pre-condizioni}
\pcell{1}{
		Nessuna.
}
\pline
\ptitle{Flusso}
\pcell{1}{
		\begin{enumerate} \item Il cliente di HBS naviga fino alla pagina di gestione dei conti correnti, dopo aver effettuato il login;
\item il cliente di HBS seleziona un conto corrente fra quelli che possiede presso la banca;
\item il sistema calcola saldo contabile, attuale e liquido per il conto corrente selezionato;
\item l'informazione calcolata viene mostrata al cliente di HBS. \end{enumerate}
}
\pline
\ptitle{Post-condizioni}
\pcell{1}{
		Nessuno.
}
\pline
\ptitle{Side effects}
\pcell{1}{
		Nessuno.
}
\end{ptable}

