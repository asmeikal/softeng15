\subsection{\code{UC\_1} - Creazione Operazione Veloce }
\label{sec:use-case:CROPVEL}

\begin{ptable}{2}
\ptitle{Titolo}
\pcell{1}{
	Creazione Operazione Veloce (\code{UC\_CROPVEL\_1})
}
\pline
\ptitle{Descrizione use case}
\pcell{1}{
		I dipendenti della banca devono poter creare delle operazioni veloci specificando il metodo di traduzione dei parametri delle operazioni veloci nei parametri dell'operazione tradizionale corrispondente.
}
\pline
\ptitle{Attori}
\pcell{1}{
		Dipendente della banca.
}
\pline
\ptitle{Origine}
\pcell{1}{
		Requisiti funzionali (requisito di sistema \idCROPVEL, sez.~\ref{req:sec:sistema:funzionali:CROPVEL} del documento dei requisiti).
}
\pline
\ptitle{Pre-condizioni}
\pcell{1}{
		Le operazioni veloci sono state abilitate dal manager della banca.
}
\pline
\ptitle{Flusso}
\pcell{1}{
		\begin{enumerate} \item Il dipendente specifica il nome dell'operazione veloce;
\item il dipendente seleziona il tipo di operazione tradizionale in cui l'operazione veloce deve essere convertita;
\item per ciascun parametro dell'operazione tradizionale il dipendente specifica un parametro dell'operazione veloce con relativo dominio e meccanismo (algoritmo) di traduzione, o una costante;
\item \label{itm:CROPVEL:last} il dipendente salva l'operazione veloce. \end{enumerate}
}
\pline
\ptitle{Post-condizioni}
\pcell{1}{
		Nessuna.
}
\pline
\ptitle{Side effects}
\pcell{1}{
		L'operazione veloce creata dal dipendente della banca \`e stata memorizzata nel sistema di HBS.
}
\pline
\ptitle{Flusso alternativo}
\pcell{1}{
		Prima del punto~\ref{itm:CROPVEL:last} il dipendente pu\`o annullare l'operazione senza provocare cambiamenti al sistema.
}
\pline
\ptitle{Post-condizioni alternative}
\pcell{1}{
		Nessuna.
}
\pline
\ptitle{Side effects alternativi}
\pcell{1}{
		Nessuno.
}
\end{ptable}

\subsection{\code{UC\_2} - Disposizione di Pagamento }
\label{sec:use-case:DISPAG}

\begin{ptable}{2}
\ptitle{Titolo}
\pcell{1}{
	Disposizione di Pagamento (\code{UC\_DISPAG\_2})
}
\pline
\ptitle{Descrizione use case}
\pcell{1}{
		Un cliente di HBS deve poter effettuare disposizioni di pagamento generiche, quali bonifici e bonifici SEPA.
Le disposizioni di pagamento necessitano di ulteriore autenticazione tramite OTP.
}
\pline
\ptitle{Attori}
\pcell{1}{
		Cliente di HBS.
}
\pline
\ptitle{Origine}
\pcell{1}{
		Requisiti funzionali (requisito di sistema \idDISPAG, sez.~\ref{req:sec:sistema:funzionali:DISPAG} del documento dei requisiti).
}
\pline
\ptitle{Pre-condizioni}
\pcell{1}{
		Il cliente di HBS ha i permessi per effettuare operazioni dal conto corrente selezionato.
}
\pline
\ptitle{Flusso}
\pcell{1}{
		\begin{enumerate} \item Il cliente di HBS seleziona il tipo di operazione da effettuare (bonifico, bonifico SEPA, etc);
\item il cliente compila i campi necessari per effettuare l'operazione (somma di denaro, data di valuta dell'operazione, conto corrente del destinatario, etc);
\item il cliente invia il form compilato al sistema di HBS;
\item il sistema di HBS richiede OTP al cliente di HBS;
\item \label{itm:dispag:last} il cliente inserisce la OTP;
\item \label{itm:conferma} il sistema conferma la presa in carico dell'operazione da parte del back-end della banca. \end{enumerate}
}
\pline
\ptitle{Post-condizioni}
\pcell{1}{
		L'operazione indicata dall'utente \`e stata presa in carico con successo dal back-end della banca.
}
\pline
\ptitle{Side effects}
\pcell{1}{
		Nessuno.
}
\pline
\ptitle{Flusso alternativo}
\pcell{1}{
		L'utente pu\`o interrompere l'operazione in qualunque momento prima del punto~\ref{itm:dispag:last} senza che questa abbia conseguenze.
Nel caso la OTP ricevuta dal sistema al punto~\ref{itm:conferma} non sia corretta, o nel caso in cui il back-end rifiuti l'operazione, lo use case procede con il flusso alternativo.
}
\pline
\ptitle{Post-condizioni alternative}
\pcell{1}{
		Nessuna.
}
\pline
\ptitle{Side effects alternativi}
\pcell{1}{
		Nessuno.
}
\end{ptable}

\subsection{\code{UC\_3} - Iscrizione Correntista }
\label{sec:use-case:ISCRCORR}

\begin{ptable}{2}
\ptitle{Titolo}
\pcell{1}{
	Iscrizione Correntista (\code{UC\_ISCRCORR\_3})
}
\pline
\ptitle{Descrizione use case}
\pcell{1}{
		Un cliente della banca non registrato ad HBS deve poter effettuare la pre-registrazione in remoto.
}
\pline
\ptitle{Attori}
\pcell{1}{
		Cliente della banca.
}
\pline
\ptitle{Origine}
\pcell{1}{
		Requisiti funzionali (requisito di sistema \idISCRCORR, sez.~\ref{req:sec:sistema:funzionali:ISCRCORR} del documento dei requisiti).
}
\pline
\ptitle{Pre-condizioni}
\pcell{1}{
		Il cliente della banca non \`e registrato a HBS, ossia non \`e un cliente di HBS.
}
\pline
\ptitle{Flusso}
\pcell{1}{
		\begin{enumerate} \item Il cliente della banca inserisce il proprio codice fiscale e il proprio numero di conto corrente nella pagina di iscrizione;
\item il sistema prende in carico la richiesta di iscrizione. \end{enumerate}
}
\pline
\ptitle{Post-condizioni}
\pcell{1}{
		Nessuna.
}
\pline
\ptitle{Side effects}
\pcell{1}{
		La richiesta di iscrizione \`e stata presa in carico dal sistema di HBS.
}
\end{ptable}

\subsection{\code{UC\_4} - Visualizzazione Bidding }
\label{sec:use-case:BIDVIS}

\begin{ptable}{2}
\ptitle{Titolo}
\pcell{1}{
	Visualizzazione Bidding (\code{UC\_BIDVIS\_4})
}
\pline
\ptitle{Descrizione use case}
\pcell{1}{
		I manager della banca possono ottenere delle visualizzazioni delle regole di bidding realizzate. Una visualizzazione fornisce una rappresentaziona grafica intuitiva dell'insieme di bid approvati automaticamente, soggetti ad approvazione da parte di un manager, e respinti automaticamente.
}
\pline
\ptitle{Attori}
\pcell{1}{
		Manager della banca
}
\pline
\ptitle{Origine}
\pcell{1}{
		Requisiti funzionali e requisiti di usabilit\`a.
}
\pline
\ptitle{Pre-condizioni}
\pcell{1}{
		Almeno una regola di bidding \`e stata definita.
}
\pline
\ptitle{Flusso}
\pcell{1}{
		\begin{enumerate} \item Il manager seleziona una regola di bidding definita in HBS; \item il sistema di HBS realizza una visualizzazione della regola di bidding; \item la visualizzazione prodotta viene mostrata al manager nel browser. \end{enumerate}
}
\pline
\ptitle{Post-condizioni}
\pcell{1}{
		Nessuna
}
\pline
\ptitle{Side effects}
\pcell{1}{
		Nessuno
}
\end{ptable}

\subsection{\code{UC\_5} - Disposizione Operazione Veloce }
\label{sec:use-case:DISOPVEL}

\begin{ptable}{2}
\ptitle{Titolo}
\pcell{1}{
	Disposizione Operazione Veloce (\code{UC\_DISOPVEL\_5})
}
\pline
\ptitle{Descrizione use case}
\pcell{1}{
		Un cliente di HBS deve poter effettuare le operazioni veloci configurate dalla banca.
}
\pline
\ptitle{Attori}
\pcell{1}{
		Cliente di HBS.
}
\pline
\ptitle{Origine}
\pcell{1}{
		Requisiti funzionali (requisito di sistema \idDISOPVEL, sez.~\ref{req:sec:sistema:funzionali:DISOPVEL} del documento dei requisiti).
}
\pline
\ptitle{Pre-condizioni}
\pcell{1}{
		Almeno una tipologia di operazione veloce \`e stata configurata dai dipendenti della banca.
}
\pline
\ptitle{Flusso}
\pcell{1}{
		\begin{enumerate} \item L'utente seleziona la tipologia di operazione veloce che desidera effettuare;
\item l'utente inserisce i parametri richiesti dall'operazione veloce;
\item l'utente invia l'operazione veloce al sistema di HBS;
\item \label{itm:DISOPVEL:validazione} il sistema di HBS valida e traduce l'operazione veloce in operazione ordinaria;
\item il sistema di HBS richiede la OTP al cliente di HBS;
\item \label{itm:DISOPVEL:conferma-cliente} il cliente di HBS invia al sistema la OTP;
\item \label{itm:DISOPVEL:otp} il sistema di HBS valida la OTP;
\item \label{itm:DISOPVEL:back-end} il sistema di HBS invia l'operazione ordinaria risultante dalla traduzione dell'operazione veloce al sistema di OBP;
\item il sistema di HBS restituisce all'utente conferma dell'avenuta presa in carico da parte del back-end della banca dell'operazione ordinaria corrispondente all'operazione veloce richiesta. \end{enumerate}
}
\pline
\ptitle{Post-condizioni}
\pcell{1}{
		Nessuna.
}
\pline
\ptitle{Side effects}
\pcell{1}{
		L'operazione ordinaria corrispondente all'operazione veloce richiesta dall'utente \`e stata presa in carico correttamente dal back-end della banca.
}
\pline
\ptitle{Flusso alternativo}
\pcell{1}{
		Il cliente di HBS pu\`o interrompere l'operazione in qualunque momento prima del punto~\ref{itm:DISOPVEL:conferma-cliente} senza produrre risultati. Il sistema pu\`o restituire un errore ai punti~\ref{itm:DISOPVEL:validazione}, \ref{itm:DISOPVEL:otp} e \ref{itm:DISOPVEL:back-end}, nel caso in cui, rispettivamente, i parametri dell'operazione veloce non siano validi, la OTP non sia valida, e il back-end rifiuti l'operazione ordinaria risultante. In questo caso non viene prodotto alcun risultato e l'utente visualizza una pagina di errore.
}
\pline
\ptitle{Post-condizioni alternative}
\pcell{1}{
		Nessuna.
}
\pline
\ptitle{Side effects alternativi}
\pcell{1}{
		Nessuno.
}
\end{ptable}

\subsection{\code{UC\_6} - Bidding Utente }
\label{sec:use-case:USRBID}

\begin{ptable}{2}
\ptitle{Titolo}
\pcell{1}{
	Bidding Utente (\code{UC\_USRBID\_6})
}
\pline
\ptitle{Descrizione use case}
\pcell{1}{
		Un utente di HBS pu\`o effettuare bid per conti correnti, carte di credito o prestiti.
}
\pline
\ptitle{Attori}
\pcell{1}{
		Utente di HBS
}
\pline
\ptitle{Origine}
\pcell{1}{
		Requisiti funzionali (requisito utente~\ref{req:itm:utente:funzionali:bidding:utente}, sez.~\ref{req:sec:utente:funzionali}, e requisito di sistema \idUSRBID).
}
\pline
\ptitle{Pre-condizioni}
\pcell{1}{
		Il sistema di bidding \`e stato attivato dalla banca.
}
\pline
\ptitle{Flusso}
\pcell{1}{
		\begin{enumerate} \item l'utente seleziona l'oggetto per cui desidera fare bid fra quelli per cui l'istituto permette operazioni di bid;
\item l'utente specifica i parametri del proprio bid;
\item l'utente invia il bid al sistema di HBS;
\item l'utente riceve un responso dal sistema di HBS riguardo l'approvazione, il rifiuto o la presa in carico del bid da parte del sistema stesso. \end{enumerate}
}
\pline
\ptitle{Post-condizioni}
\pcell{1}{
		Nessuna.
}
\pline
\ptitle{Side effects}
\pcell{1}{
		Se la proposta di bid \`e stata rifiutata, nessuno. Se la proposta di bid \`e stata accettata, il bid \`e stato memorizzato nel sistema come bid approvato automaticamente, e inviato a un dipendente della banca per la realizzazione. Se la proposta di bid \`e stata presa in carico per il controllo da parte di un manager della banca, la proposta di bid \`e stata inserita nel sistema come proposta non ancora approvata, ed \`e stata inviata al manager della banca.
}
\pline
\ptitle{Flusso alternativo}
\pcell{1}{
		In qualsiasi punto del workflow l'utente pu\`o interrompere l'inserimento della proposta di bid.
}
\pline
\ptitle{Post-condizioni alternative}
\pcell{1}{
		Nessuna.
}
\pline
\ptitle{Side effects alternativi}
\pcell{1}{
		Nessuno.
}
\end{ptable}

\subsection{\code{UC\_7} - Creazione Bidding }
\label{sec:use-case:CREABID}

\begin{ptable}{2}
\ptitle{Titolo}
\pcell{1}{
	Creazione Bidding (\code{UC\_CREABID\_7})
}
\pline
\ptitle{Descrizione use case}
\pcell{1}{
		Un manager della banca pu\`o creare una regola di bidding.
}
\pline
\ptitle{Attori}
\pcell{1}{
		Manager della banca.
}
\pline
\ptitle{Origine}
\pcell{1}{
		Requisiti funzionali (requisito di sistema \idCREABID).
}
\pline
\ptitle{Pre-condizioni}
\pcell{1}{
		Nessuna.
}
\pline
\ptitle{Flusso}
\pcell{1}{
		\begin{enumerate} \item Il manager seleziona una tipologia di bidding (carte di credito, conto corrente, prestito); \item \label{itm:uc:CREABID:parametro} il manager seleziona un parametro di scelta; \item il manager imposta i range di approvazione automatica, approvazione condizionata, e rifiuto automatico sul parametro scelto; \item opzionalmente, il manager torna al punto \ref{itm:uc:CREABID:parametro}; \item il manager inserisce la regola di bidding creata. \end{enumerate}
}
\pline
\ptitle{Post-condizioni}
\pcell{1}{
		Nessuna.
}
\pline
\ptitle{Side effects}
\pcell{1}{
		La regola di bidding realizzata \`e stata aggiunta al sistema di HBS.
}
\pline
\ptitle{Flusso alternativo}
\pcell{1}{
		In qualsiasi punto del workflow il manager pu\`o annullare l'operazione.
}
\pline
\ptitle{Post-condizioni alternative}
\pcell{1}{
		Nessuna.
}
\pline
\ptitle{Side effects alternativi}
\pcell{1}{
		Nessuno.
}
\end{ptable}

