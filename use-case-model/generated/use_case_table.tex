\subsection{\code{UC\_1} - Visualization Bidding }
\label{sec:use-case:BIDVIS}

\begin{ptable}{2}
\ptitle{Titolo}
\pcell{1}{
	Visualization Bidding (\code{UC\_BIDVIS\_1})
}
\pline
\ptitle{Descrizione use case}
\pcell{1}{
		I manager della banca possono ottenere delle \emph{visualization} delle regole di bidding realizzate. Una \emph{visualization} fornisce una rappresentaziona grafica intuitiva dell'insieme di bid approvati automaticamente, soggetti ad approvazione da parte di un manager, e respinti automaticamente.
}
\pline
\ptitle{Attori}
\pcell{1}{
		Manager della banca
}
\pline
\ptitle{Origine}
\pcell{1}{
		Requisiti funzionali e requisiti di usabilit\`a.
}
\pline
\ptitle{Pre-condizioni}
\pcell{1}{
		Almeno una regola di bidding \`e stata definita.
}
\pline
\ptitle{Flusso}
\pcell{1}{
		\begin{enumerate} \item Il manager seleziona una regola di bidding definita in HBS; \item il sistema di HBS realizza una \emph{visualization} della regola di bidding; \item la \emph{visualization} prodotta viene mostrata al manager nel browser. \end{enumerate}
}
\pline
\ptitle{Post-condizioni}
\pcell{1}{
		Nessuna
}
\pline
\ptitle{Side effects}
\pcell{1}{
		Nessuno
}
\end{ptable}

\subsection{\code{UC\_2} - Creazione Bidding }
\label{sec:use-case:CREABID}

\begin{ptable}{2}
\ptitle{Titolo}
\pcell{1}{
	Creazione Bidding (\code{UC\_CREABID\_2})
}
\pline
\ptitle{Descrizione use case}
\pcell{1}{
		Un manager della banca pu\`o creare una regola di bidding.
}
\pline
\ptitle{Attori}
\pcell{1}{
		Manager della banca.
}
\pline
\ptitle{Origine}
\pcell{1}{
		Requisiti funzionali (requisito di sistema \idCREABID).
}
\pline
\ptitle{Pre-condizioni}
\pcell{1}{
		Nessuna.
}
\pline
\ptitle{Flusso}
\pcell{1}{
		\begin{enumerate} \item Il manager seleziona una tipologia di bidding (carte di credito, conto corrente, prestito); \item \label{itm:uc:CREABID:parametro} il manager seleziona un parametro di scelta; \item il manager imposta i range di approvazione automatica, approvazione condizionata, e rifiuto automatico sul parametro scelto; \item opzionalmente, il manager torna al punto \ref{itm:uc:CREABID:parametro}; \item il manager inserisce la regola di bidding creata. \end{enumerate}
}
\pline
\ptitle{Post-condizioni}
\pcell{1}{
		Nessuna.
}
\pline
\ptitle{Side effects}
\pcell{1}{
		La regola di bidding realizzata \`e stata aggiunta al sistema di HBS.
}
\pline
\ptitle{Flusso alternativo}
\pcell{1}{
		In qualsiasi punto del workflow il manager pu\`o annullare l'operazione.
}
\pline
\ptitle{Post-condizioni alternative}
\pcell{1}{
		Nessuna.
}
\pline
\ptitle{Side effects alternativi}
\pcell{1}{
		Nessuno.
}
\end{ptable}

