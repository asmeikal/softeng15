\section{Analisi dei costi}

% TODO scrivere analisi dei costi

La stima dei costi di realizzazione del progetto viene effettuata con la tecnica degli \emph{Use Case Points}.

Gli Use Case Points del progetto sono dati dalla somma di due addendi:
\begin{description}
	\item[Unadjusted Use Case Weight]
		Misura il numero e la complessit\`a degli Use Case del sistema.
	\item[Unadjusted Actor Weight]
		Misura il numero e la complessit\`a degli attori del sistema.
\end{description}
Il valore ottenuto dalla somma di \code{UUCW} e \code{UAW} viene moltiplicato per due fattori:
\begin{description}
	\item[Technical Complexity Factor]
		Misura la complessit\`a tecnica del sistema.
	\item[Environmental Complexity Factor]
		Misura la complessit\`a dell'ambiente in cui il sistema si inserir\`a.
\end{description}

\subsection{Unadjusted Use Case Weight}

Il \code{UUCW} assegna un punteggio a ciascun use case del sistema secondo la complessit\`a dello use case.

Il \code{UUCW} sar\`a determinato a individuazione degli use case ultimata.

\subsection{Unadjusted Actor Weight}

Il \code{UAW} sar\`a determinato a individuazione degli use case ultimata.

\subsection{Technical Complexity Factor}

\begin{tabularx}{\columnwidth}{c X c c}
\toprule
\cellcolor{color2!10} Fatt. & \cellcolor{color2!10} Descrizione & \cellcolor{color2!10} Peso & \cellcolor{color2!10} Val. \\
\midrule
\code{T1} & Sistema distribuito & $2.0$ & --- \\
\code{T2} & Obiettivi di performance/tempo di risposta & $1.0$ & --- \\
\code{T3} & Efficienza lato utente & $1.0$ & --- \\
\code{T4} & Complessit\`a processi interni & $1.0$ & --- \\
\code{T5} & Riusabilit\`a del codice & $1.0$ & --- \\
\code{T6} & Facilit\`a d'installazione & $0.5$ & 3 \\
\code{T7} & Facilit\`a d'utilizzo & $0.5$ & 4 \\
\code{T8} & Portabilit\`a inter-piattaforma & $2.0$ & --- \\
\code{T9} & Manutenzione del sistema & $1.0$ & --- \\
\code{T10} & Parallelismo/concorrenza & $1.0$ & --- \\
\code{T11} & Sicurezza & $1.0$ & 5 \\
\code{T12} & Accesso terze parti & $1.0$ & 5 \\
\code{T13} & Addestramento utente & $1.0$ & 0 \\
\midrule
\multicolumn{3}{r}{Totale \code{TCF}} & --- \\
\bottomrule
\end{tabularx}

\subsection{Environmental Complexity Factor}

\begin{tabularx}{\columnwidth}{c X r r}
\toprule
\cellcolor{color2!10} Fatt. & \cellcolor{color2!10} Descrizione & \cellcolor{color2!10} Peso & \cellcolor{color2!10} Val. \\
\midrule
\code{E1} & Familiarit\`a con processo software adottato & $1.5$ & --- \\
\code{E2} & Esperienza nel dominio dell'applicazione & $0.5$ & --- \\
\code{E3} & Esperienza del team con orientazione a oggetti & $1.0$ & --- \\
\code{E4} & Capacit\`a primo analista & $0.5$ & --- \\
\code{E5} & Motivazione del team & $1.0$ & --- \\
\code{E6} & Stabilit\`a dei requisiti & $2.0$ & --- \\
\code{E7} & Staff part-time & $-1.0$ & --- \\
\code{E8} & Linguaggio di programmazione ostico & $-1.0$ & --- \\
\midrule
\multicolumn{3}{r}{Totale \code{ECF}} & --- \\
\bottomrule
\end{tabularx}

\subsection{Use Case Points e previsione di costo}

Gli Use Case Points saranno determinati a individuazione degli use case ultimata.
